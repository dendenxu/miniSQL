\documentclass[UTF8]{ctexrep} % Chinese language class type, need XeLaTeX to compile
\usepackage{hologo} % needed for reference
\usepackage{gbt7714} % needed for reference
\usepackage{float} % needed for [H] strick floating option
\usepackage{caption} % needed for command \captionsetup in longlisting new environment
\usepackage{subcaption} % needed for command \captionsetup in longlisting new environment
\usepackage{graphicx} % needed for displaying eps file
\usepackage{longtable} % to display tables on several pages
\usepackage{rotating} % to display tables in landscape
\usepackage{multirow} % for multirow tables
\usepackage{booktabs} % for prettier tables
\usepackage[title]{appendix} % create appendices
%-----------------------------------------------------------------------------------------
% To write pseudo-code
%-----------------------------------------------------------------------------------------
\usepackage{algorithm}  
\usepackage{algpseudocode}
\usepackage{amsmath}
\usepackage{amssymb}
\renewcommand{\algorithmicrequire}{\textbf{Input:}}  % Use Input in the format of Algorithm  
\renewcommand{\algorithmicensure}{\textbf{Output:}} % Use Output in the format of Algorithm
%-----------------------------------------------------------------------------------------
% To adjust geometry and stuff
%-----------------------------------------------------------------------------------------
\usepackage{geometry} % adjust margin and what
\geometry{a4paper} % use A4 paper to make the report look good enough
%-----------------------------------------------------------------------------------------
% To define colors yourself
%-----------------------------------------------------------------------------------------
\usepackage{xcolor} % needed for defining own color
% \definecolor{myred}{RGB}{223, 56, 27}
% \definecolor{myblue}{RGB}{25, 123, 193}
% \definecolor{myorange}{RGB}{225, 107, 65}
% \definecolor{mypink}{RGB}{254, 63, 125}
% \definecolor{mycyan}{RGB}{72, 199, 240}
% \definecolor{mygreen}{RGB}{0, 153, 136}
%-----------------------------------------------------------------------------------------
% To use hyper-reference and stuff
%------- ----------------------------------------------------------------------------------
\usepackage{hyperref} % needed for hyperlinks
\hypersetup{ % hyper link setups, can be used else where
    colorlinks=true,
    linkcolor=blue,
    filecolor=magenta,      
    urlcolor=cyan,
    pdftitle={imageRestore},
}
%-----------------------------------------------------------------------------------------
% To list code elegantly and stuff
%-----------------------------------------------------------------------------------------
\usepackage[
    % cache=false,
    newfloat=true,
    % outputdir=./
]{minted} % needed for listing
\newenvironment{longlisting}{\captionsetup{type=listing}}{}
\setminted{
    tabsize=4,
    breaklines,
    frame=single,
    linenos,
    fontsize=\small
}


\makeatletter
\newenvironment{breakablealgorithm}
  {% \begin{breakablealgorithm}
   \begin{center}
     \refstepcounter{algorithm}% New algorithm
     \hrule height.8pt depth0pt \kern2pt% \@fs@pre for \@fs@ruled
     \renewcommand{\caption}[2][\relax]{% Make a new \caption
       {\raggedright\textbf{\ALG@name~\thealgorithm} ##2\par}%
       \ifx\relax##1\relax % #1 is \relax
         \addcontentsline{loa}{algorithm}{\protect\numberline{\thealgorithm}##2}%
       \else % #1 is not \relax
         \addcontentsline{loa}{algorithm}{\protect\numberline{\thealgorithm}##1}%
       \fi
       \kern2pt\hrule\kern2pt
     }
  }{% \end{breakablealgorithm}
     \kern2pt\hrule\relax% \@fs@post for \@fs@ruled
   \end{center}
  }
\makeatother

% \algrenewtext{For}[3]%
% {\algorithmicfor\ #1 \gets #2 \algorithmicto\ #3 \algorithmicdo}


%-----------------------------------------------------------------------------------------
% To predefine title and stuff
%-----------------------------------------------------------------------------------------
\title{数据库系统实验报告\\
  B+ Tree and Index Manager in \texttt{miniSQL} \\
  \large \texttt{miniSQL}设计中的B+树和索引管理器}

\author{
教师:孙建伶\\
姓名:徐震\\
学号:3180105504\\
邮箱:\href{3180105504@zju.edu.cn}{3180105504@zju.edu.cn}\\
电话:18888916826
}
\date{\today}

\begin{document}

\maketitle
\newpage
\tableofcontents
\newpage
\part{正文}
\chapter{实验目的}
\paragraph{\texttt{miniSQL}}
设计并实现一个精简型单用户SQL引擎(DBMS)\texttt{miniSQL},允许用户通过字符界面输入SQL 语句实现表的建立/删除;索引的建立/删除以及表记录的插入/删除/查找。
\paragraph{索引管理器}
我们负责设计的索引管理器模块主要负责数据库系统中的索引管理,提供基于B+树的索引实现并提高数据库查询/插入/删除效率。并通过提供易用接口与其他模块整合实现有效功能。
\paragraph{设计目的}
通过对\texttt{miniSQL}的设计与实现,提高学生的系统编程能力,加深对数据库系统原理的理解。
通过编程设计,加深对数据库系统的理解并深入了解B+树这一数据结构。
以模块化方式构建大型计算机软件,提高架构抽象能力并重视模块化和解耦合在软件设计中的作用。
\paragraph{功能实现}
我们将实现如下的基本索引功能:
\begin{itemize}
    \item 对单属性索引提供完整支持。
    \item 通过客制化比较函数可支持简单多属性索引。
    \item 对于表的主键自动建立B+树索引/排序数组索引。
    \item 对于声明为\texttt{unique}(唯一值)的属性可以通过SQL语句由用户指定建立/删除B+树索引。
    \item 支持除B+树以外的数据结构索引(如排序数组索引),控制接口。
    \item 基于\texttt{Exception}(异常)的错误信息传递路径。
    \item 基于\texttt{Python}语言的多种数据类型索引支持。
\end{itemize}
\chapter{实验环境}
\section{系统需求}
\paragraph{主要开发语言}
\begin{itemize}
    \item Python 3.7.5
    \item Python 3.7.6
\end{itemize}
\paragraph{主要开发环境}
\begin{itemize}
    \item PyCharm 2020.1
    \item Visual Studio Code 1.45
\end{itemize}
\paragraph{经过测试的系统环境}
\begin{itemize}
    \item Microsoft Windows [Version 10.0.18363.836]
    \item Ubuntu WSL2
\end{itemize}

\section{实验环境}
\paragraph{实验系统环境} Microsoft Windows [Version 10.0.18363.836]
\paragraph{实验处理器环境} Intel(R) Core(TM) i7-9750H CPU @ 2.60GHz 12 Logical Processors
\paragraph{实验内存环境} SODIMM 15.8GB/16.0GB
\paragraph{实验硬盘环境} KBG30ZMS512G NVMe TOSHIBA 512GB

\chapter{模块设计}
\section{功能描述}
\subsection{建立/删除索引}
实际的数据库应用中,有在建立表时创建索引和表中有数据情况下创建索引的需求。区别在于,前者不需要其他模块为索引管理器提供数据,仅仅需要分配一个新索引所需的内存和磁盘空间;而后者要求发出建立索引请求的模块提供相应的数据块。默认情况下我们以记录的行号作为B+树中键值对上的“值”,因此我们要求相应数据块是按照它们将来被查找的顺序提供的。
\paragraph{建立索引}
正如图\ref{fig:index_example}所示的\footnote{图\ref{fig:index_example}来源于\textit{Database System Concepts 6th Edition Abraham Silberschatz等。}}。为了方便演示,让我们用其中的第一列作为建立索引的数据集。我们会按顺序将提供的数据插入索引数据结构,因此在这一个键值对中,键为第一列数据值,如\textit{10101},而值为行号,如\textit{0}。
\begin{figure}[H]
    \centering
    \includegraphics[width=0.7\linewidth]{figure/index_example.eps}
    \caption{线性记录储存方式}
    \label{fig:index_example}
\end{figure}

这样处理的原因在于,我们希望日后通过索引根据查找键快速找到记录对应的位置,而其在表格中的相对位置是最方便的寻址信息之一。若我们按照索引的自定义值储存插入的键,则索引失去加快搜索的作用;若我们插入的值为记录的绝对磁盘位置信息,则缓存管理器失效,且数据的移动会导致索引失效。
\par
在内存中创建索引(并插入相应数据后),索引管理器会将索引的内存信息叫给缓存管理器,由其决定是否应将内存保留或者存储到磁盘中,同时返回给索引管理器一个唯一的索引标号(索引管理器会将其继续返回给上层模块),日后将根据这一唯一标识符从缓存管理器中取得相应索引(无论是通过读取磁盘文件还是直接获取内存指针)。
\paragraph{删除索引}
我们通过上述的唯一标识符给缓存管理器发出删除信号,完成删除操作。

\subsection{查找/删除索引键}
查找和删除操作支持快速范围操作,并且两者在具体实现上有极大相似性,我们通过抽象两者的操作来提高代码复用率。
我们首先直接判断用户进行的是范围还是单值查找,并且替前从缓存管理器中取得相应索引内容(内存或硬盘中)。
\paragraph{单值操作}
我们调用B+树相应查找接口获得应查找的值,并通过异常来进行错误通讯。若查找成功则直接返回,否则抛出相关异常。
\begin{figure}[H]
    \centering
    \includegraphics[width=0.75\linewidth]{figure/search_delete_single.eps}
    \caption{单值操作}
    \label{fig:search_delete_single}
\end{figure}


\paragraph{范围操作}
我们首先会检查用户给予的范围是否有效\footnote{例如,范围左右下标是否为左 小右大,或被查找的树是否为空等},同样的,我们使用异常来进行错误通讯,这使得接口模块能方便的实现错误处理。接着我们查询范围两端的值\footnote{我们默认范围查找的区间是左闭右开的。}并根据返回的节点情况进行相关操作,对于查找指令,这一操作是返回查询得到的相关信息;对于删除指令,这一操作是操作B+树删除相关的值,并在操作全部完成后将修改后的索引叫给缓存管理器。如图\ref{fig:search_delete_multi}所示(加深部分为需要查找或删除的部分)。

\begin{figure}[H]
    \centering
    \includegraphics[width=0.75\linewidth]{figure/search_delete_multi.eps}
    \caption{范围操作}
    \label{fig:search_delete_multi}
\end{figure}

\subsection{插入键值对}
我们单独实现了插入键值对功能,因为其逻辑相对于查询和删除操作都有所不同(需检查重复元素等)。类似的,索引管理器会首先向缓存管理器请求相关索引内容。接着我们调用插入操作,将是否允许替换的信息传递给B+树的相关寒函数直接进行操作\footnote{原因在于B+树内部实现中也需要调用查找相关功能,与其规定执行流程,不如让B+树具体实现获得最优执行流程}。

\subsection{更新值内容}
由于我们使索引中键值对中\textit{值}指向记录在某张表中的位置,而这一位置在数据库运行过程中可能会发生很大变化\footnote{例如闲时的数据库清理和记录重排序等,亦或记录管理器采取了不同的数据储存模型。},我们给外界留出批量修改值的接口。
\par
这一操作类似于范围操作中涉及到的内容,我们需要对B+中的有效节点数目进行检查,根据节点数目选取不同的处理方式。

\section{主要数据结构}
\subsection{B+树}
\footnote{此段内容摘自\href{https://en.wikipedia.org/wiki/B+_tree}{WikiPedia}}
B+ 树是一种树数据结构,通常用于数据库和操作系统的文件系统中。B+ 树的特点是能够保持数据稳定有序,其插入与修改拥有较稳定的对数时间复杂度。B+ 树元素自底向上插入,这与二叉树恰好相反。
\par
B+ 树在节点访问时间远远超过节点内部访问时间的时候,比可作为替代的实现有着实在的优势。这通常在多数节点在次级存储比如硬盘中的时候出现。通过最大化在每个内部节点内的子节点的数目减少树的高度,平衡操作不经常发生,而且效率增加了。这种价值得以确立通常需要每个节点在次级存储中占据完整的磁盘块或近似的大小。
\par
B+ 背后的想法是内部节点可以有在预定范围内的可变量目的子节点。因此,B+ 树不需要像其他自平衡二叉查找树那样经常的重新平衡。对于特定的实现在子节点数目上的低和高边界是固定的。例如,在 2-3 B 树(常简称为2-3 树)中,每个内部节点只可能有 2 或 3 个子节点。如果节点有无效数目的子节点则被当作处于违规状态。
\par
B+ 树的创造者 Rudolf Bayer 没有解释B代表什么。最常见的观点是B代表平衡(balanced),因为所有的叶子节点在树中都在相同的级别上。B也可能代表Bayer,或者是波音(Boeing),因为他曾经工作于波音科学研究实验室。

\paragraph{查找}
查找以典型的方式进行,类似于二叉查找树。起始于根节点,自顶向下遍历树,选择其分离值在要查找值的任意一边的子指针。在节点内部典型的使用是二分查找来确定这个位置。

\paragraph{插入}
节点要处于违规状态,它必须包含在可接受范围之外数目的元素。
首先,查找要插入其中的节点的位置。接着把值插入这个节点中。
如果没有节点处于违规状态则处理结束。
如果某个节点有过多元素,则把它分裂为两个节点,每个都有最小数目的元素。在树上递归向上继续这个处理直到到达根节点,如果根节点被分裂,则创建一个新根节点。为了使它工作,元素的最小和最大数目典型的必须选择为使最小数不小于最大数的一半。

\paragraph{删除}
首先,查找要删除的值。接着从包含它的节点中删除这个值。
如果没有节点处于违规状态则处理结束。
如果节点处于违规状态则有两种可能情况:
它的兄弟节点,就是同一个父节点的子节点,可以把一个或多个它的子节点转移到当前节点,而把它返回为合法状态。如果是这样,在更改父节点和两个兄弟节点的分离值之后处理结束。
亦或,它的兄弟节点由于处在低边界上而没有额外的子节点。在这种情况下把两个兄弟节点合并到一个单一的节点中,而且我们递归到父节点上,因为它被删除了一个子节点。持续这个处理直到当前节点是合法状态或者到达根节点,在其上根节点的子节点被合并而且合并后的节点成为新的根节点。

\subsection{排序数组}
本数据结构是为了配合主键而实现的,采用最普通的排序数组查找方式,并在数组内部采用二分查找进行相关操作。
可从无限子树数目的B+树抽象的到,因此我们可以较为方便的统一两者的接口。
\par
值得注意的是,为了配合主键和记录管理器中数据的储存方式,我们往往使用一种特殊的类作为排序数组的内部容器:一种返回当前下标的特殊数组\footnote{我们可以利用这一特性而使得这种储存不占用任何空间,而同时保证接口的一致性。}。

\section{类图与类间关系}
\subsection{B+树}
B+树的类图实现如图\ref{fig:bplus_diagram}所示\footnote{为了类图完整性,我们也列举了除用户自定义类型以外的类型。}\footnote{我们使用矢量图渲染了字体,若图表过小请放大查看。}。
\begin{figure}[H]
    \centering
    \includegraphics[width=0.85\linewidth]{figure/bplus.eps}
    \caption{B+树的类图与类间关系}
    \label{fig:bplus_diagram}
\end{figure}
\subsection{异常类型}
程序使用的异常类图如\ref{fig:exceptions_diagram}所示。
\begin{figure}[H]
    \centering
    \includegraphics[width=0.5\linewidth]{figure/exceptions.eps}
    \caption{异常的类图与类间关系}
    \label{fig:exceptions_diagram}
\end{figure}
\subsection{索引管理器}
在具体的索引管理器实现上,我们采用了模块层面上的抽象而非类层面的,这更贴合\texttt{Python}语言的风格。
这样能保证尽量大的抽象层次与代码复用率。值得注意的是,我们在实现B+树相关操作的时候也使用了静态函数来抽象部分内容\footnote{我们将在下一部分详细阐述实现细节}。



\chapter{遇到的问题及解决方法}
\section{更改目录结构后\texttt{Python}无法正确引用其他\texttt{.py}文件}
\subsection{问题描述}
为了使文件结构更为整洁方便,我们尝试了使用模块来封装各个\texttt{.py}文件与直接调用\texttt{.py}文件或相对引用等多种方式,最初的文件目录如列表\ref{lst:filetree0}所示:
\begin{longlisting}
    \begin{minted}{shell}
        ./
        |-- some_data_folders
        |-- some_data_files
        |-- bplus.py
        |-- buffer.py
        |-- exceptions.py
        |-- ...
        |-- index.py
        |-- readme.md
    \end{minted}
    % guess you'll be needing more than the correct content but also the correct sequence to achieve proper label
    \caption{可能的文件目录}
    \label{lst:filetree0}
\end{longlisting}
这种方式下\texttt{Python}脚本只需要使用普通的\texttt{import}或\texttt{from ... import ...}即可正确引用相关文件。
\par
且得益于IDE特性\footnote{\texttt{Visual Studio Code},\texttt{PyCharm}等},当前目录会被自动设置为程序的运行时的根目录,也就意味着程序中对于文件的操作也可正常进行(即使\texttt{Python}脚本与数据文件不在同一文件夹下)。因此我们采用了如列表\ref{lst:filetree1}所示的文件组织结构。

\begin{longlisting}
    \begin{minted}{shell}
        ./
        |-- some_data_folders
        |-- some_data_files
        |-- code/
            |-- bplus.py
            |-- buffer.py
            |-- exceptions.py
            |-- ...
            |-- index.py
        |-- readme.md
        |-- license
    \end{minted}
    % guess you'll be needing more than the correct content but also the correct sequence to achieve proper label
    \caption{可能的文件目录}
    \label{lst:filetree1}
\end{longlisting}

这种文件组织结构下,code中的代码可以正常访问根目录中的数据文件和数据文件夹,但无法引用同在\texttt{code}文件夹下的其他\texttt{.py}文件。如列表\ref{lst:cannot_import}所示。
\begin{longlisting}
    \begin{minted}{shell}
In [1]: import blablabla
---------------------------------------------------------------------------
ModuleNotFoundError                       Traceback (most recent call last)
<ipython-input-1-7e2f26fa0a8b> in <module>
----> 1 import blablabla

ModuleNotFoundError: No module named 'blablabla'

In [2]:
    \end{minted}
    % guess you'll be needing more than the correct content but also the correct sequence to achieve proper label
    \caption{无法引用同一文件下的模块}
    \label{lst:cannot_import}
\end{longlisting}

\subsection{解决方法}
我们注意到如下事实:
\begin{itemize}
    \item \texttt{Python}根据\texttt{PYTHONPATH}设定的目录进行文件,模块等的搜索。
    \item PyCharm会将当前工作目录的根目录当作\texttt{Python}工作目录之一,即添加到\texttt{PYTHONPATH}中,如列表\ref{lst:path_append}所示。
    \item Visual Studio Code会根据当前工作目录下的环境配置文件(\texttt{.env})而进行目录与模块搜索。
\end{itemize}

因此,在PyCharm中我们可以通过修改目录等级,对目录做标记来手动使IDE搜索相关内容,如图\ref{fig:folder_label}所示。在Visual Studio Code中我们可以通过编辑如列表\ref{lst:manual_append}所示的环境配置文件来添加相关内容。
\par
由于上述两种方法都是在间接修改运行时的\texttt{PYTHONPATH},在只用Python执行相关脚本时我们可以通过如列表\ref{lst:path_append}所示的语句执行等价操作。

\begin{figure}[H]
    \centering
    \includegraphics[width=0.4\linewidth]{figure/path_append.jpg}
    \caption{PyCharm中的文件夹标记}
    \label{fig:folder_label}
\end{figure}


\begin{longlisting}
    \begin{minted}{python}
D:\condaenvs\Anaconda\python.exe "C:\Program Files\JetBrains\PyCharm 2019.3.4\plugins\python\helpers\pydev\pydevconsole.py" --mode=client --port=4243
import sys; print('Python %s on %s' % (sys.version, sys.platform))
sys.path.extend(['D:\\Documents\\Materials\\DENDEN\\PROJ\\Database\\miniSQL', 'D:\\Documents\\Materials\\DENDEN\\PROJ\\Database\\miniSQL\\code', 'D:/Documents/Materials/DENDEN/PROJ/Database/miniSQL']) 
    \end{minted}
    \caption{PyCharm中的搜索目录自动修改}
    \label{lst:path_append}
\end{longlisting}


\begin{longlisting}
    \begin{minted}{python}
PYTHONPATH=code
    \end{minted}
    \caption{PyCharm中的搜索目录自动修改}
    \label{lst:manual_append}
\end{longlisting}

\section{使用\texttt{pickle}储存二进制形式的索引超出递归层数限制}
树是一个递归实现的数据结构,\texttt{Python}提供的二进制储存包\texttt{pickle}需要通过递归调用树中的内容来储存其所包含的信息。因此树高过大会导致超出\texttt{Python}递归限制,如图\ref{fig:recursion_limit}所示。

\begin{figure}[H]
    \centering
    \includegraphics[width=0.4\linewidth]{figure/recursion_limit.png}
    \caption{超出递归限制}
    \label{fig:recursion_limit}
\end{figure}

当然我们可以使用列表\ref{lst:change_recursion_limit}所示的代码修改\texttt{Python}对于递归深度的限制。但修改词限制之后系统堆栈同样有机会溢出,在这种情况下系统会直接杀死我们的\texttt{Python}进程,如图\ref{fig:kill_process}所示。

\begin{figure}[H]
    \centering
    \includegraphics[width=0.4\linewidth]{figure/system_termination.png}
    \caption{系统杀死当前进程}
    \label{fig:kill_process}
\end{figure}

\begin{longlisting}
    \begin{minted}{python}
import sys
sys.setrecursionlimit(10000000)
    \end{minted}
    \caption{修改递归限制}
    \label{lst:change_recursion_limit}
\end{longlisting}
最终,我们通过修改B+树的最大子树数量避免了这一问题,在调用\texttt{create\_index}方法时直接设置M值,如列表\ref{lst:change_m}所示。

\begin{longlisting}
    \begin{minted}{python}
def create_index(ind, data_list, cmp=dummy_cmp, is_primary=False):
    """
    :param ind: the id of the index to be saved to file
    :param data_list: the data, as list, to create index on
    :param cmp: the comparator of the index, defaults to operator<
    :param is_primary: whether we're dealing with primary key, using sorted list
    :return: index of the newly created table
    """
    if is_primary:
        t = SortedList()
    else:
        # TODO: dynamically compute the M value of the B+ tree
        # TODO: what if you're out of memory
        t = Tree(m=50, cmp=cmp)
    for index, data in enumerate(data_list):
        # TODO: what happens if you get an error from the B+ tree
        t.insert(data, index)  # insert data as key and line number as value
    \end{minted}
    \caption{调整M值,降低深度}
    \label{lst:change_m}
\end{longlisting}

\chapter{总结}
在本实验中,我们从零实现了B+树模块。借助\texttt{Python}的动态类型语言特性,我们的B+树可以有效对各种\texttt{Python}支持的数据类型进行操作。通过进行单元测试,我们在miniSQL项目整体进行的前期就保证了B+树模块的正确性,为后续调试整合过程带来了极大方便。
\par
接着,我们实现了B+树中的索引相关操作。并封装成了一个易用的模块,使得API可以方便的调用。
\par
此次实验中我们深刻感受到了小组合作与提前定义接口的重要性,并且在B+树具体实现过程中,我们对单元测试的作用有了更进一步的了解。


\newpage
\part{附录}


\begin{appendices}
\chapter{接口说明}

\chapter{插图,表格与列表}
\listoffigures
% \listoftables
\listoflistings
\end{appendices}
\end{document}